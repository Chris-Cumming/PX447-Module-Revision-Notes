%----------------------------------------------------------------------------------------
% DOCUMENT CONFIGURATIONS
%----------------------------------------------------------------------------------------

\documentclass[
10pt, % The default document font size, options: 10pt, 11pt, 12pt
oneside, % Two side (alternating margins) for binding by default, uncomment to switch to one side
english, % other languages available
% singlespacing, % Single line spacing, alternatives: onehalfspacing or doublespacing
onehalfspacing, % Single line spacing, alternatives: onehalfspacing or doublespacing
%draft, % Uncomment to enable draft mode (no pictures, no links, overfull hboxes indicated)
%nolistspacing, % If the document is onehalfspacing or doublespacing, uncomment this to set spacing in lists to single
%liststotoc, % Uncomment to add the list of figures/tables/etc to the table of contents
%toctotoc, % Uncomment to add the main table of contents to the table of contents
]{McMasterThesis} % The class file specifying the document structure


%----------------------------------------------------------------------------------------
% Import packages here
%----------------------------------------------------------------------------------------
\usepackage[utf8]{inputenc} % Required for inputting international characters
\usepackage[T1]{fontenc} % Output font encoding for international characters

\usepackage{lmodern} % could change font type by calling a different package
\usepackage{lastpage} % count pages
\usepackage{siunitx} % for scientific units (micro-liter, etc)
\usepackage{xlop} %For arithmetic operations in a table.

\setcounter{tocdepth}{2} % so that only section and sub sections appear in Table of Contents. Remove or set depth to 3 to include sub-sub-sections


\usepackage{braket}
\usepackage{amsmath}



\usepackage{rotating} % rotating table and figure
\usepackage[T1]{fontenc}
\usepackage[utf8]{inputenc}
 
%% Text box -----------------------
\usepackage{mdframed} % text box
% all 4 borders
\newmdenv{allfour}

% just top and bottom
\newmdenv[leftline=false,rightline=false]{topbot}

% just left and bottom
\newmdenv[topline=false,rightline=false]{leftbot}


%% Stylist Text Boxe
\usepackage[many]{tcolorbox}
% \definecolor{main}{HTML}{5989cf}    % setting main color to be used
% \definecolor{sub}{HTML}{cde4ff}     % setting sub color to be used

\tcbset{
    sharp corners,
    % colback = white,
    before skip = 0.2cm,    % add extra space before the box
    after skip = 0.5cm      % add extra space after the box
}                           % setting global options for tcolorbox
\newtcolorbox{boxK}{
    center,
    width=5.2in,
    % height=0.9in,
    % sharpish corners, % better drop shadow
    boxrule = 0pt,
    toprule = 0pt, % top rule weight
    % enhanced,
    % fuzzy shadow = {0pt}{-2pt}{-0.5pt}{0.5pt}{black!35} % {xshift}{yshift}{offset}{step}{options} 
}

%% Blind tex
	

\usepackage{blindtext}

%----------------------------------------------------------------------------------------
% Chapter wise Citations
%----------------------------------------------------------------------------------------
\usepackage[sectionbib]{natbib}
\usepackage{chapterbib}
%%% Uncomment if you change the bibliography heading/title
\renewcommand{\bibname}{References}

%%% Uncomment if you want to include the bibliographies at the end of each chapter in the table of contents.  
\usepackage[nottoc]{tocbibind}


%----------------------------------------------------------------------------------------
% Collect all your header information from the chapters here, things like acronyms, custom commands, necessary packages, etc. 
%----------------------------------------------------------------------------------------
\usepackage{parskip} %this will put spaces between paragraphs
\setlength{\parindent}{15pt} % this will create and indent on all but the first paragraph of each section. 
% should maybe change to glossaries package
\usepackage{acro}
\DeclareAcronym{est}{
	short = EST,
	long  = expressed sequence tags
}

\DeclareAcronym{Xl}{
	short = \textit{X.~laevis},
	long  = \textit{Xenopus~laevis}
}
\DeclareAcronym{Xg}{
	short = \textit{X.~gilli},
	long  = \textit{Xenopus~gilli}
}

\usepackage{etoolbox}
\preto\chapter{\acresetall} % resets acronyms for each chapter

\usepackage{xspace} %helps spacing with custom commands. 
\newcommand{\oddname}{{\sc SoME goOfY LonG ThiNg With an AwkWarD NAme}\xspace}


\usepackage{pgfplotstable} % a much better way to handle tables
\pgfplotsset{compat=1.12}





%----------------------------------------------------------------------------------------
%	THESIS INFORMATION
%----------------------------------------------------------------------------------------

%% Thesis Title (Choose only one) --------------
% \thesistitle{Big Data Clustering Models and Applications in Research and Subscription-based Platforms} % Your thesis title, print it elsewhere with \ttitle

% \thesistitle{Large-Scale Data Clustering Models with Applications in Research and Subscription-based Platforms} % Your thesis title, print it elsewhere with \ttitle

% \thesistitle{Big Data Clustering Models with Applications in Research and Subscription-based Invoicing Platforms} % Your thesis title, print it elsewhere with \ttitle

%\thesistitle{Collection of Quantum Information Notes} % Your thesis title, print it
%% -------------------------------



%\supervisor{IBM} % Your supervisor's name, print it elsewhere with \supname
%\examiner{} % Your examiner's name, print it elsewhere with \examname
%\degree {Self Study} % Your degree name, print it elsewhere with \degreename
%\shortdegree{}
%\subjectarea{Quantum Information} % Your degree area name, print it elsewhere with \area
%\author{Chris Cumming} % Your name, print it elsewhere with \authorname
%\addresses{} % Your address, print it elsewhere with \addressname
%\subject{Physics} % Your subject area, print it elsewhere with \subjectname
%\keywords{} % Keywords for your thesis, print it elsewhere with \keywordnames
%\university{\href{}} % Your university's name and URL, print it elsewhere with \univname
%\department{\href{}} % Your department's name and URL, print it elsewhere with \deptname
%\group{\href{}{}} % Your research group's name and URL, print it elsewhere with \groupname
%\faculty{\href{}{}} % Your faculty's name and URL, print it elsewhere with \facname

% this sets up hyperlinks
\hypersetup{pdftitle=\ttitle} % Set the PDF's title to your title
\hypersetup{pdfauthor=\authorname} % Set the PDF's author to your name
\hypersetup{pdfkeywords=\keywordnames} % Set the PDF's keywords to your keywords



%----------------------------------------------------------------------------------------
% Begin document
%----------------------------------------------------------------------------------------
\begin{document}
\frontmatter

\frontmatter % Use roman page numbering style (i, ii, iii, iv...) for the pre-content pages

\pagestyle{plain} % Default to the plain heading style until the thesis style is called for the body content

%----------------------------------------------------------------------------------------
%	Half Title (lay title). 
%   60 Characters with spaces
%----------------------------------------------------------------------------------------
%\begin{halftitle} % could not get this environment working
%\vspace*{\fill}
\vspace*{2.5in}
\begin{center}
\LARGE \textsc{PX447: Quantum Computation, Simulation and Information}% ideally, but it doesn't seem to matter
\end{center}
%\vspace*{\fill}
\vfill
\pagenumbering{gobble} % leave this here, McMaster doesn't want this page numbered
%\end{halftitle}
%\clearpage

%----------------------------------------------------------------------------------------
%	TITLE PAGE
%----------------------------------------------------------------------------------------
\pagenumbering{gobble}
\begin{center}

\vfill
\textsc{\Large \ttitle} \\

\vspace{50pt}


\vfill
%By\\
%\uppercase\expandafter{\authorname},\\
%M.Sc. (Computer Science)\\%% -----> List prior degrees after comma  <----

\vspace{50pt}

 \vfill
{\large}\\

\vspace{30pt}

\vfill
%{\large  \degreename}\\ % \vspace{15pt}
%{\large  in}\\ %\vspace{15pt}
%{\large  \subjectareaname}\\
\vfill
\vfill

\vspace{50pt}

%{\large \univname\, \\ Hamilton, Ontario}\\[4cm] % replace \today with the submission date


%{\copyright\, Copyright by \authorname,\, \today}\\[4cm] % replace \today with the submission date

\end{center}

%----------------------------------------------------------------------------------------
%	Descriptive note numbered ii
%----------------------------------------------------------------------------------------
% Need to add below info
\newpage
\pagenumbering{roman} % leave to turn numbering back on
\setcounter{page}{2} % leave here to make this page numbered ii, a Grad School requirement

\noindent % stops indent on next line
%\degreename\, (\the\year) \\
%\deptname \\
%\univname \\ 
%Hamilton, Ontario, Canada \\[1.5cm]
%TITLE: \ttitle 
\\ \\
%AUTHOR:\\ \authorname, \\
%M.Sc. (Computer Science)\,\\  %list previous degrees  
\\ \\
%SUPERVISOR:\\ 
%\supname\,\\
%Professor, Department or School Name, \\
%McMaster University, ON, Canada
\\ 
\\
%SUPERVISORY COMMITTEE CHAIR: \\ 
%Dr. XXX XXXX,\\
%Professor, Department or School Name, \\
%McMaster University, ON, Canada
\\
\\
%SUPERVISORY COMMITTEE MEMBERS: \\Dr. XXX XXXX\\ 
%Professor, Department or School Name, \\
%McMaster University, ON, Canada
\\
\\
%Dr. XXX XXXXX\\ 
%Professor, Department or School Name, \\
%McMaster University  
\\
\\
%NUMBER OF PAGES: \pageref{lastoffront}, \pageref{LastPage}  % put in iv and number

%\clearpage



%----------------------------------------------------------------------------------------
%	ABSTRACT PAGE
%----------------------------------------------------------------------------------------
\vspace{-10cm}
\section*{\Huge Introduction} 
%\addchaptertocentry{Introduction}
% Type your abstract here. 
It should be noted that throughout this module there is a mixture of wave mechanics and matrix mechanics. You may not have realised it but so far you've dealt pretty exclusively with wave mechanics. These are just different ways of performing calculations in quantum mechanics and the end result for both is equivalent, so you shouldn't worry too much about the detail behind it. All you should really worry about is that the formalism in wave mechanics is based off partial differential equations and wave theory such that the operators are derivatives acting on wave functions, and is normally used when dealing with infinite dimensional systems. Whilst matrix mechanics is much more handy when dealing with finite dimensional systems where we can use linear algebra to solve problems more easily, as the operators in this case are now represented by matrices acting on state vectors. Dirac was the one to show that these 2 formalisms are equal. He also came up with the bra-ket notation which is arguably the more fundamental way of forming problems in quantum mechanics, as it doesn't specify if we are using wave or matrix mechanics. In general throughout the module bra-ket notation is used with Martin's section using a mixture of wave and matrix mechanics, whereas Ben's section uses only matrix mechanics.

\noindent Another thing to note is that again so far you have only really been working in the Schrodinger picture. However there is also the Dirac/interaction picture and the Heisenberg picture, which are used in this module and later ones. These different pictures correspond to where we put the time dependence in quantum mechanics. In the Schrodinger picture, the time dependence is put in the eigenstates and the operators have no time dependence. Whilst in the Heisenberg picture we put the time dependence in the operators and the eigenstates are time independent. Like with wave mechanics and matrix mechanics these provide the same predictions as each other, its just in some cases choosing one over the other makes the maths simpler. The Dirac picture combines these together, such that some time dependence is in the operators and some in the eigenstates.
%\clearpage
%----------------------------------------------------------------------------------------
%	ACKNOWLEDGEMENTS
%----------------------------------------------------------------------------------------

% \begin{acknowledgements}
% \addchaptertocentry{\acknowledgementname} % Add the acknowledgements to the table of contents

% The acknowledgements and the people to thank go here, don't forget to include your project adviser\ldots

% \end{acknowledgements}

%\section*{\Huge \centering Acknowledgements}
%\addchaptertocentry{Acknowledgements}
%\vspace{20pt}

\hspace{\parindent}


%\clearpage

%----------------------------------------------------------------------------------------
%	LIST OF CONTENTS/FIGURES/TABLES PAGES
%----------------------------------------------------------------------------------------

\tableofcontents % Prints the main table of contents

%\listoffigures % Prints the list of figures

%\listoftables % Prints the list of tables


%----------------------------------------------------------------------------------------
%	DECLARATION PAGE
%----------------------------------------------------------------------------------------
%\begin{declaration}
% \addchaptertocentry{\authorshipname}

%\noindent I, \authorname, declare that this thesis titled, \textbf{\ttitle}, and %works presented in it are my own. I confirm that 

%\begin{itemize} 
%\item List each chapter
%\item and what you have done for it
%\end{itemize}
 
%\end{declaration}

%----------------------------------------------------------------------------------------
% The following bit is just here to make sure we end up on a new page and get the total number of roman numeral
\label{lastoffront}
%\clearpage
% make sure this command is on the last of your frontmatter pages, i.e. only this command, a \clearpage then \mainmatter
% should be fine without modification
%----------------------------------------------------------------------------------------

%----------------------------------------------------------------------------------------
%	THESIS CONTENT - CHAPTERS
%----------------------------------------------------------------------------------------

\mainmatter % Begin numeric (1,2,3...) page numbering

\pagestyle{thesis} 

\chapter{Classical Computation}
\label{chapt1}

% remember to set these at the start of each chapter
\chapter{Reversible Classical Computation}
\label{chapt2} 


\chapter{Quantum Mechanics}
\label{chapt3}

\chapter{Quantum Computation}
\label{chapt4}


\chapter{Quantum Simulation}
\label{chapt5}



\chapter{Quantum Search}
\label{chapt6}



\chapter{Solving Linear Systems of Equations}
\label{chapt7}




\chapter{Quantum Fourier Transform}
\label{chapt8}



\chapter{Quantum Error Correction}
\label{chapt9}

So far we have focused exclusively on quantum computation and briefly on simulation. The following chapter is instead very much in the realm of quantum information.
\include{chap10}
\include{chap11}
\include{chap12}


%----------------------------------------------------------------------------------------
%	THESIS CONTENT - APPENDICES
%----------------------------------------------------------------------------------------

\appendix % Cue to tell LaTeX that the following "chapters" are Appendices
\renewcommand{\thetable}{A\arabic{chapter}.\arabic{table}} % adds an A to table names in appendix (Table A1.1, A1.2...)
\renewcommand{\thefigure}{A\arabic{chapter}.\arabic{figure}} % same for figures
\renewcommand{\thesection}{A\arabic{section}}

% Include the appendices of the thesis as separate files from the Appendices folder
% \input{Appendix/Supp_Chap3.tex}


% \backmatter
% %% A list of publications can be created using this approach
% \include{ownpubs}

\end{document}
